\section{Gaussian to Other}
\begin{enumerate}
\item
Let $X_1 \sim  \gauss{0}{1}$ and $X_2 \sim  \gauss{0}{1}$. Plot the CDF and PDF of
%
\begin{equation}
V = X_1^2 + X_2^2
\end{equation}\\
\solution Refer The CDF and PDF of $V$ plots in \figref{fig:chisq_cdf} and \figref{fig:chisq_pdf} 
\begin{lstlisting}
	chapter4/codes/sos.py
	chapter4/codes/sos1.py
\end{lstlisting}
\begin{figure}[H]
\centering
\includegraphics[scale=0.8]{chapter4/figs/chisq_cdf.pdf}
\caption{CDF of $V$}
\label{fig:chisq_cdf}
\end{figure}
\begin{figure}[H]
\centering
\includegraphics[scale=0.8]{chapter4/figs/chisq_pdf.pdf}
\caption{PDF of $V$}
\label{fig:chisq_pdf}
\end{figure}
%

%
%
\item
If
%
\begin{equation}
F_{V}(x) = 
\begin{cases}
1 - e^{-\alpha x} & x \geq 0 \\
0 & x < 0,
\end{cases}
\label{eq:chisq2_cdf_gen}
\end{equation}
%
find $\alpha$.\\
\solution 
%
\begin{lstlisting}
	chapter4/codes/cdf6.py
\end{lstlisting}
\begin{figure}
\centering
\includegraphics[scale=0.8]{chapter4/figs/cdf6.pdf}
\caption{The CDF of $V$ for different alpha}
\label{fig:gauss_cdf_alpha}
\end{figure}
from \ref{fig:gauss_cdf_alpha}
$alpha$= 0.5
\item
\label{ch3_raleigh_sim}
Plot the CDF and PDF of
%
\begin{equation}
A = \sqrt{V}
\end{equation}\\
\solution 
The CDF of $A$ is given by,
\begin{align}
	F_{A}\brak{a} &= \pr{A < a}\\
	&= \pr{\sqrt{V} < a}\\
	&= \pr{V < a^2}\\
	&= F_{V}\brak{a^2}\\
	&= 1-\exp\brak{-\frac{a^2}{2}} 
\end{align}
Using \eqref{eq:cdf_to_pdf}, the PDF is found to be
\begin{align}
	p_{A}\brak{a} &= a\exp\brak{-\frac{a^2}{2}}
\end{align}
\begin{lstlisting}
	chapter4/codes/cpdf.py
	chapter4/codes/cpdf1.py
\end{lstlisting}
\begin{figure}[H]
\centering
\includegraphics[scale = 0.8]{chapter4/figs/rayleigh_cdf.pdf}
\caption{CDF of $A$}
\label{fig:rayleigh_cdf}
\end{figure}
\begin{figure}[H]
\centering
\includegraphics[scale=0.8]{chapter4/figs/rayleigh_pdf.pdf}
\caption{PDF of $A$}
\label{fig:rayleigh_pdf}
\end{figure}
%
\end{enumerate}

\section{Conditional Probability}
\begin{enumerate}
\item
\label{ch4_sim}
Plot 
\begin{equation}
P_e = \pr{\hat{X} = -1|X=1}
\end{equation}
%
for 
\begin{equation}
Y = AX+N,
\end{equation}
where $A$ is Raleigh with $E\sbrak{A^2} = \gamma, N \sim \gauss{0}{1}, X \in \brak{-1,1}$ for $0 \le \gamma \le 10$ dB.\\
\solution Refer \figref{fig:bpsk_pe_snr_rayleigh} 
\begin{lstlisting}
	chapter4/codes/cp.py
\end{lstlisting}
%
\item
Assuming that $N$ is a constant, find an expression for $P_e$.  Call this $P_e(N)$\\
\solution \solution The estimated value $\hat{X}$ is given by
\begin{align}
\hat{X} = 
\begin{cases}
+1 & Y>0\\
-1 & Y<0
\end{cases}
\end{align}
For $X = 1$, 
\begin{align}
Y &= A + N\\
P_e &= \pr{\hat{X} = -1|X=1} \\
&= \pr{Y<0 |X=1}\\
&= \pr{A<-N}\\
&= F_A(-N)\\
&= \int_{-\infty}^{-N} f_A(x)dx
\end{align}
By definition
\begin{align}
f_A(x) = 
\begin{cases}
\frac{x}{\sigma^2}\exp\brak{{-\frac{x^2}{2\sigma^2}}} & x\geq0\\
0 & otherwise
\end{cases}
\end{align}
If $N>0, f_A(x) = 0$. Then,
\begin{align}
 P_e=0  
\end{align}
If $N<0$. Then,
\begin{align}
 P_e(N) &=\int_{-\infty}^{-N} f_A(x)dx\\
 &=\int_{-\infty}^{0} 0dx+\int_{0}^{-N} f_A(x)dx\\
 &=\int_{0}^{-N} \frac{x}{\sigma^2}\exp\brak{{-\frac{x^2}{2\sigma^2}}}dx\\
 &=1-\exp{\brak{-\frac{N^2}{2\sigma^2}}}
\end{align}
Therefore,
\begin{align}\label{pe(N)}
P_e(N) = 
\begin{cases}
1-\exp\brak{{-\frac{N^2}{2\sigma^2}}} & N<0\\
0 & otherwise
\end{cases}
\end{align}
%
\item
%
\label{ch4_anal}
For a function $g$,
\begin{equation}
E\sbrak{g(X)} = \int_{-\infty}^{\infty}g(x)p_{X}(x)\, dx
\end{equation}
%
Find $P_e = E\sbrak{P_e(N)}$.\\
\solution Since $N \sim \gauss{0}{1}$ ,
\begin{align}
  p_N(x)= \frac{1}{\sqrt{2\pi}}\exp \brak{-\frac{x^2}{2} }
\end{align}
And from \eqref{pe(N)} 
\begin{align}
    P_e(x)=
    \begin{cases}
1-\exp\brak{{-\frac{x^2}{2\sigma^2}}} & x<0\\
0 & otherwise
\end{cases}
\end{align}

\begin{align}
 P_e=E\sbrak{P_e(N)} = \int_{-\infty}^{\infty}P_e(x)p_{N}(x)\, dx  
\end{align}
If $x<0, P_e(x)=0$ and using the fact that for an even function
\begin{align}
\int_{-\infty}^{\infty}f(x)=2\int_{-\infty}^{0}f(x)   
\end{align}
we get
\begin{align}
  P_e&= \frac{1}{\sqrt{2\pi}}\int_{-\infty}^{0}\exp \brak{ -\frac{x^2}{2}} \brak{1-\exp \brak{ -\frac{x^2}{2\sigma^2}} } dx\\
&= \frac{1}{2\sqrt{2\pi}} \int_{-\infty}^{\infty} \exp \brak{ -\frac{x^2}{2} }dx \nonumber \\
&- \frac{1}{2\sqrt{2\pi}} \int_{-\infty}^{\infty} \exp \brak{-\frac{(1+ \sigma^2)x^2}{2 \sigma^2}}  dx\\
&= \frac{\sqrt{2\pi} - \sqrt{\frac{\pi(2\sigma^2)}{1+\sigma^2}}}{2\sqrt{2\pi}}\\
&= \frac{1}{2} - \frac{1}{2}\sqrt{\frac{\sigma^2}{1+\sigma^2}}
\end{align}
For a Rayleigh Distribution with scale $= \sigma$,
\begin{align}
E\sbrak{A^2} = 2\sigma^2\\
\gamma = 2\sigma^2\\
\therefore P_e = \frac{1}{2} - \frac{1}{2}\sqrt{\frac{\gamma}{2+\gamma}}
\end{align}
%
\item
Plot $P_e$ in problems \ref{ch4_sim} and \ref{ch4_anal} on the same graph w.r.t $\gamma$.  Comment.\\
\solution $P_e$ plotted in same graph in \figref{fig:bpsk_pe_snr_rayleigh}. The value of $P_e$ is much higher when the channel %
gain $A$ is Rayleigh distributed than the case where $A$ is a constant (compare with \figref{fig:bpsk_pe_snr}).
\begin{figure}[H]
\centering
\includegraphics[scale=0.8]{chapter4/figs/prob_error.pdf}
\caption{$P_e$ versus $\gamma$}
\label{fig:bpsk_pe_snr_rayleigh}
\end{figure}
\end{enumerate}
